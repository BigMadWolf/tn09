\chapter{Conclusion}

Ce stage chez \asl\ m'a au final beaucoup plu. En effet, passionné de technologie, les différentes missions de développement web qui m'ont été confiées ont parfaitement répondu à mes attentes en terme de challenge technique. Ayant déjà développé avec \asf~1.0 un an et demi auparavant, j'ai pu me rapproprier l'outil et apprendre à me servir de sa nouvelle mouture, estampillée 1.3. J'ai eu également l'opportunité de découvrir de nouvelles technologies, comme le \ajs, \ajquery, \aajax, ainsi que de nouveaux concepts, comme celui des \awss\ par exemple. Enfin, j'ai pu renforcer ma connaissance du modèle \amvc\ et améliorer ma façon de versionner mon code source avec \asvn. J'ai pu acquérir toutes ces compétences grâce aux conseils et au \acoaching\ des membres de l'équipe de développement de \asl.

Au delà du domaine technique, ce stage m'a permis de découvrir le professionnalisme dans le milieu du développement informatique. J'ai pu acquérir des méthodologies issues l'expérience accumulée dans l'entreprise, que je pourrai remettre en pratique dans mon futur cadre professionnel voire dès mon retour à l'\autc. En outre, participer à des réunions de retour d'expérience ou de remise en question de méthodes de travail a été très enrichissant, dans le sens où cela développe notre esprit critique vis-à-vis de notre activité.

Si je dois toutefois trouver une déception vis-à-vis de mon expérience chez \asl, je dirais que c'est de ne pas avoir eu l'occasion de participer à l'effort communautaire \aos\ autour de \asf. En effet, ce n'est pas une tâche qui peut être comprise dans le temps de travail d'un développeur : lors de l'entretien précédent mon stage, mon responsable \ahugon\ m'avait prévenu. Ainsi, j'ai bien baigné dans un environnement utilisant de nombreuses technologies \aos, mais je n'ai pas été un acteur de l'évolution de celles-ci.

Pour conclure, ce stage d'assistant ingénieur a fait murir en moi quelques idées de projet professionnel. Par exemple, la configuration des environnements d'exécution des applications web sur les serveurs les hébergeant, qui correspond à un premier niveau de l'administration système, est une activité qui m'a plu : j'ai d'ailleurs choisi la filière SRI pour continuer ma formation à l'\autc. Aussi, développer une application métier telle que l'\aintranet\ d'\aey, qui sera utilisée par des dizaines d'utilisateurs novices en informatique, m'a fait réfléchir sur des problématiques d'ergonomie : c'est un thème qui m'intéresse beaucoup et pour lequel il serait possible d'exercer un métier de consultant. Enfin, les domaines du génie logiciel et de l'écosystème du libre me passionnent toujours autant. Je profiterai donc de mes derniers semestres de formation de l'\autc\ pour laisser décanter ces expériences et ensuite préciser mon projet.
