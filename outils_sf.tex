\subsection{\asf}

\asf\footnote{\url{http://www.symfony-project.org/}} est le \afm\ \aphp\ dont l'initiateur est \apotencier, un des co-fondateurs de \asl. Dans un premier temps conservé en interne dans l'agence, il a finalement été distribué sous licence libre en octobre~2005. Depuis, une véritable communauté internationale s'est formée autour de cet outil. Au delà des qualités techniques intrinsèques de \asf, on peut expliquer ce phénomène par le fait que qu'un effort considérable a été fourni concernant l'écriture de documentation : d'abord rédigée en anglais, elle a ensuite été traduite en plusieurs langues. De plus, plusieurs livres techniques\footnote{Références bibliographiques : \cite{practicalsf} \cite{sfrefguide} \cite{cahierssf} \cite{moresf} \cite{thebook}} ont déjà été publiés, dont la plupart voient leur contenu accessible librement et de façon officielle sur \ainternet.

Les parties suivantes abordent successivement une présentation du concept de \afm, du modèle \amvc, qui est utilisé de façon globale dans \asf, et enfin une énumération des différentes fonctionnalités de \asf.


\subsubsection{Le concept de \afm}

TODO


\subsubsection{Le modèle \amvc}

TODO


\subsubsection{Fonctionnalités de \asf}

TODO
