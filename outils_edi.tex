\subsection{Environnements de développement}

Chez \asl, le choix de l'environnement de développement est donné au développeur. Certains utilisent des éditeurs de texte comme \aultraedit\ ou \avim, d'autres des environnements de développement intégré\footnote{Un environnement de développement intégré (EDI ou IDE en anglais) est un programme regroupant un ensemble d'outils pour le développement de logiciels.\cite{edi}} tels que \aeclipse\ ou \anetbeans.

Pour ma part, j'ai choisi \anetbeans\footnote{\url{http://www.netbeans.org/}}. En effet, j'étais déjà bien habitué à cet environnement, que j'utilisais pour programmer en \ajava\footnote{J'ai d'ailleurs participé au développement du logiciel \agephi\ (\url{http://www.gephi.org/}) qui repose sur la même base que \anetbeans, appelée \anbp.} Il supporte nativement de nombreux langages de programmation, dont notamment \aphp. Il a pu me fournir, entre autres, les facilités suivantes :

\begin{itemize}
	\item une coloration syntaxique des différents formats de fichier supportés par \asf, tels que \aphp, \ahtml\ ou \ayml ;
	\item une auto-complétion de code adaptée à \asf ;
	\item un support intégré de \asvn\footnote{Cf. section\ref{section:outils_svn}}.
\end{itemize}
