\subsection{Contexte et objectifs}
\label{section:eyrolles_contexte}

Le groupe \aey\ est un groupe français d'édition spécialisé dans les domaines du livre professionnel et technique, et publie notamment des livres consacrés à l'informatique.

Historiquement, \aey\ gérait l'ensemble de ses données par le biais de documents sous format papier, d'autres sous format numérique (documents \amsword, \amsexcel\dots), ainsi que des \aemails. Utiliser de tels supports en tant que base de référence pour l'ensemble des processus métiers engendre des inconvénients majeurs :

\begin{itemize}
	\item les mêmes informations sont ressaisies plusieurs fois dans les documents ;
	\item les informations ne sont pas toujours saisies de la même manière ;
	\item la mise à jour des informations n'est pas toujours répercutée partout ;
	\item l'information circule sous des formats hétérogènes, d'où une difficulté de réutilisation.
\end{itemize}

\aey\ a alors pris la décision de constituer une véritable base d'information accessible sous la forme d'un \aintranet\ par les acteurs des différents services. Une partie de ce projet a été mis en production sous l'appellation de \emph{\alotun}. Celle-ci propose une base de données produits, associée entre autres aux fonctionnalités suivantes :

\begin{itemize}
	\item la gestion des processus métiers spécifiques au service commercial ;
	\item la création de documents, à partir de modèles prédéfinis, permettant de créer des documents commerciaux ;
	\item l'export d'information en fonction de critères choisis par l'utilisateur (format, type d'informations).
\end{itemize}

Exploité depuis mars~2007, ce \alotun\ est en constante évolution.

Dans la continuité, le groupe \aey\ souhaite étendre l'\aey\ afin de disposer d'une base de projets éditoriaux commune, fiable et accessible par l'ensemble des services. Cette nouvelle base serait associée à une interface de gestion et de suivi des processus d'édition, de fabrication, de commercialisation et de
promotion des livres.

L'agence web \asl\ a alors été choisie en tant que prestataire pour analyser puis implémenter ce nouveau projet, intitulé \emph{\alotdeux}.
