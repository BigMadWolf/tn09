\subsection{Organisation du travail}
\label{section:eyrolles_organisation}

Pour ma part, j'ai été affecté en tant que développeur sur le projet \aey. J'ai pu travailler avec \acohen, le chef de projet, ainsi que les développeurs \ahamon, \aweistroff\ et \abachelet. 

Cette section décrit les différentes phases du projet \aey\ et reprend son planning.


\subsubsection{Phase de spécification}

Conjointement avec le client, \acohen\ s'est chargé de la rédaction du cahier des charges du projet. Il en ressort un dossier d'une centaine de pages, décrivant les fonctionnalités attendues, les règles métiers ou encore des prototypes d'interface utilisateur. C'est ce document qui a servi de fil rouge tout au long de la réalisation du projet.


\subsubsection{Phase d'analyse}

La phase d'analyse du projet a consisté à mettre en place le modèle de l'application sous la forme d'un diagramme, le \emph{\amcd}\footnote{Le \amcd, ou Modèle Conceptuel de Données, établit sous forme de diagramme une représentation des données et définit les dépendances fonctionnelles entre elles.}. À l'\autc, nous avons eu l'occasion de réaliser des \amcds\ dans l'\auv\ NF17\footnote{Cours de conception de bases de données à l'\autc.} en utilisant le format \auml\footnote{\auml\ (\textit{Unified Modeling Language}) est un langage de modélisation graphique standardisé utilisé dans le domaine du génie logiciel, notamment dans le cadre de la conception orientée objet.\cite{uml}} ou \aea\footnote{Le modèle entité-association, ou modèle \aea, est une représentation abstraite et et conceptuelle de données. Il est notamment utilisé pour modéliser des schémas de base de données.\cite{ea}}. Chez \asl, les diagrammes \amcds\ sont réalisés au format \aea\ et sont réalisés à l'aide de l'outil \amysqlwb.

Le \amcd\ du \alotdeux\ d'\aey, modélisé par \acohen\ et \ahamon, comprend ~tables et ~liens. Pour gagner en lisibilité, les tables et les liens n'ont pas tous été représentés : seuls les plus importants permettant de mieux comprendre le fonctionnement de l'application ont été choisis.


\subsubsection{Phase de développement}

TODO


\subsubsection{Phase de recette}

TODO


\subsubsection{Livraison finale}

TODO


\subsubsection{Planning}

Le planning tel qu'il a été établi au début du projet est le suivant :

\begin{description}
	\item[Phase de développement] du 21 septembre 2009 au 11 décembre 2009
	\item[Phase de recette] du 14 décembre 2009 au 8 janvier 2010
	\item[Livraison finale] le 11 janvier 2010
\end{description}

Comme décrit précédemment, le planning réel a effectivement pris du retard sur le planning initial :

\begin{description}
	\item[Phase de développement] du 21 septembre 2009 au 11 décembre 2009
	\item[Phase de recette] du 14 décembre 2009 au 17 février 2010
	\item[Livraison finale] le 18 février 2010
\end{description}

Note : les phases de spécification et d'analyse ne sont pas listées dans le planning du fait que je n'étais pas encore arrivé chez \asl\ quand elles ont eu lieu.
