\subsection{Le langage \aphp}

\aphp\footnote{\url{http://www.php.net/}} est le langage de programmation utilisé chez \asl. C'est un langage de script libre principalement utilisé pour produire des pages web dynamiques : le code est intégré dans un document \ahtml\footnote{\ahtml\ est le format de données conçu pour représenter les pages web.~\cite{html}} puis interprété par le serveur~\ahttp\footnote{Un serveur \ahttp\ est un logiciel servant des requêtes respectant le protocole du même nom.~\cite{serveurhttp} En fait, ce sont ces types de serveurs qui envoient aux navigateurs les pages web de l'\ainternet.} doté d'un module \aphp. Il peut également être utilisé via un interpréteur lancé en ligne de commande, pour exécuter des programmes localement par exemple.

La syntaxe de \aphp\ est empruntée aux langages C, Java et Perl, afin de le rendre plus facile à apprendre. Depuis la version~5, il est capable de produire des programmes à la conception orientée objet.
