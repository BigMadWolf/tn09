\subsection{\avservers}

Chez \asl, il existe une machine relativement puissante qui est divisée en plusieurs machines virtuelles surnommées \avservers. Celles-ci ont l'avantage de pouvoir être facilement dupliquées et reviennent beaucoup moins cher que des machines physiques dédiées à la performance équivalente. 

Un \avserver\ prend ainsi l'apparence d'un serveur \alinux\ classique et in\-dé\-pen\-dant. Chacun appartient à un développeur ou à un chef de projet. Son propriétaire est le seul à pouvoir y accéder et en possède les droits administrateurs : il peut donc y installer tous les outils dont il aurait besoin.

Typiquement, le développeur se sert de son \avserver\ pour afficher les pages web qu'il développe grâce au serveur~\ahttp\ et au serveur de base de données qui y sont installés. C'est également par ce biais que les chefs de projets ou les autres développeurs peuvent consulter le fruit de son travail.

Finalement, ce système de \avservers\ est un excellent compromis entre uniformisation des environnements de développement et flexibilité.
