\section{Mini-développements}

Chez \asl, les développeurs travaillent quotidiennement sur un ou deux projets principaux. À coté de cela, ponctuellement, on leur affecte des tâches relativement courtes à réaliser qui concernent d'autres projets : ces tâches sont appelées mini-développements, ou \emph{minidevs}.

Un \aminidev\ consiste généralement à développer une fonctionnalité mineure sur un site web, à modifier légèrement un site pour lequel le client a payé le support, ou à corriger un \abug\ sur un projet sous garantie. Il est affecté de préférence à un développeur qui a déjà travaillé sur le projet concerné. 

Les paragraphes suivants décrivent les \aminidevs\ que j'ai eu à réaliser durant mon stage.

\paragraph{\alc} Comme j'avais déjà développé pour ce site, j'ai été choisi pour créer une page affichant les statistiques mensuelles des inscriptions à la \anewsletter. Cette page est accessible uniquement par les le client et le chef de projet.

\paragraph{jaicreemonentreprise.com} Ce site commandé par Infogreffe permet à de nouveaux entrepreneurs de poser leurs questions, de lire des conseils d'experts en gestion d'entreprise et à donner de la visibilité à leur propre entreprise en la décrivant. J'ai eu à changer à plusieurs reprise les bannières du site et j'ai écrit des requêtes de modification de la base de données.

\paragraph{Virbac Effipro} Virbac est un laboratoire spécialiste de la santé animale. Leur site Effipro est une plateforme multi-langues sur laquelle des vétérinaires et leurs assistants rentrent leurs données médicales, alors que les maitres des patients à quatre pattes peuvent recevoir des SMS pour leur rappeler de vacciner leur animal. Ici, j'ai eu à corriger quelques \abugs.

\paragraph{BNP Ace Manager} Ce site commandé par la BNP prend la forme d'un jeu sur \ainternet, dans lequel des étudiants internationaux issus d'écoles de commerce s'inscrivent, forment des équipes et simulent la gestion de portefeuille d'un champion de tennis. Ici aussi, mon intervention à consisté à corriger quelques \abugs.
