\section*{Résumé technique}

\asl\ est une agence web reconnue dans le monde du développement web pour avoir créé le \afm\ \aphp\ \aos\ \asf. Ses clients sont exclusivement des grands comptes, pour lesquels elle développe des applications web de taille conséquente et à la logique métier poussée.

Ayant déjà eu une première expérience intéressante avec \asf\ auparavant, je choisis \asl\ pour effectuer mon stage d'assistant ingénieur et profiter de toute l'expérience professionnelle que cela pourrait m'apporter.

Intégré à l'équipe de développement, je débute en participant à la finalisation du dernier site web des eaux de javel \alc. Je suis ensuite affecté avec d'autres développeurs à un tout nouveau projet, qui consiste à développer l'\aintranet\ des maisons d'édition \aey.

Le projet \aey\ est développé dans le langage \aphp, se base sur \asf~1.3 et communique avec une base de données \apsql\ grâce à l'\aorm\ \adoctrine. L'application est organisée en différents modules, utilise les principes de programmation orientée objet et respecte le patron de conception \amvc\ (Modèle Vue Contrôleur).

\paragraph{Mots-clés}

site web, \aphp, \afm, \asf, programmation orientée objet, \amvc, base de données, \aorm, \adoctrine