\chapter{Présentation de l'entreprise}

\section{L'entreprise \asensio}

La société \asensio\ a été créée en 1988 par les deux co-fondateurs \apotencier\ et \apascal. Dès ses premiers mois, elle s'est très vite orientée vers les technologies de l'\ainternet. Elle s'inscrit alors comme une véritable agence web interactive, proposant à ses clients un savoir-faire dans tous les métiers du web : développement, expertise technique, \awm, communication et \awd. Aujourd'hui, elle est basée dans la ville de Clichy, située près de Paris.

En 2007, \asensio\ s'est rapprochée d'\aextreme\footnote{À l'occasion du 29\ieme\ Grand Prix des \aagencesannee\ en 2009, \aextreme\ s'est vu décerner le prix du groupe de communication indépendant de l'année.}, un grand groupe in\-dé\-pen\-dant de communication globale\footnote{publicité, marketing, services, web, \textit{packaging}, design, \textit{corporate}\dots}. Cette démarche commerciale, et non capitalistique, distingue bien les deux entités en deux sociétés à part entière. Elle a pour origine un désir des deux parties : d'un côté \aextreme\ souhaitait se rapprocher d'une agence web, et de l'autre, \asensio\ ressentait le besoin d'acquérir de meilleures compétences dans le domaine de la communication.

L'association de \asensio\ et d'\aextreme\ a donné naissance à trois entités commerciales, ou \abusfull :

\begin{description}
	\item[\asl] s'occupe de la partie développement web ;
	\item[\aes] gère tout l'aspect \awm\ et communication ;
	\item[\aesm] propose à ses clients des plans média destinés à doper leur trafic et leurs ventes.
\end{description}

La force de \asensio\ réside donc dans le fait de pouvoir faire travailler ensemble des personnes aux profils de natures très différentes, et cela afin de répondre au mieux aux attentes de ses clients.

Rentable dès ses débuts, \asensio\ dégage un chiffre d'affaires de 6~millions d'euros pour l'année~2009, d'après les estimations actuelles.


\section{La \abufull\ \asl}

Chez \asensio, j'ai travaillé dans la \abufull\ \asl. Dirigée par \apotencier, son cœur de métier est de développer des applications web pour les grands comptes. En effet, ses principaux clients sont Peugeot, EDF, l'Office de Tourisme et des Congrès de Paris, Infogreffe, Evian\dots

Par ailleurs, \asl\ est le créateur d'un outil de travail connaissant aujourd'hui un succès international formidable : le \afm\ \aphp\ \asf\ décrit en section~\ref{section:outils_sf}. Celui-ci apporte à l'entreprise une visibilité grandissante associée à une image de marque. 

\asl\ est constitué de trois pôles aux ambitions différentes :

\begin{description}
	\item[le pôle projet] est composé de chefs de projets, dont le but est s'occuper de toute la partie communication avec le client, de faire l'intermédiaire entre les différents acteurs techniques d'un projet, de rédiger son cahier des charges, de planifier le temps et les ressources qui lui sont affectés ;
	\item[le pôle production] est composé d'intégrateurs, qui ont pour objectif de transformer les maquettes de sites, réalisées par les créateurs d'\aes, en pages web statiques compatibles avec un certain nombre de navigateurs du marché ;
	\item[le pôle développement] se compose d'une équipe de développeurs, dont le but est de rendre dynamiques les pages web produites par le pôle production, en se chargeant de l'implémentation de leur logique.
\end{description}

Dans le cadre de mon stage, j'ai travaillé au pôle développement en tant que développeur.

Il faut noter que certains projets, concernant des sites web évènementiels ou promotionnels par exemple, sont orchestrés par les chefs de projets d'\aes\ et non pas par ceux du pôle projet de \asl. Toutefois, les réalisations techniques sont bien issues des pôles production et développement de ce dernier.


\section{Mon choix de \asl}

Au cours de mon année de licence à l'université d'\abrookes\ en Angleterre en 2007, j'ai eu à réaliser un projet individuel et conséquent de développement logiciel. J'ai alors choisi de réaliser une application web expérimentale à mi-chemin entre un \acms\footnote{Un \acms\ (\acmsfull), ou système de gestion de contenu, est une famille de logiciels destinés à la conception et à la mise à jour dynamique de site web ou d'application multimédia.~\cite{cms}} et un réseau social\footnote{Un réseau social est un ensemble d'entités sociales, telles que des individus ou des organisations sociales, reliées entre elles par des liens créés lors d'interactions sociales.~\cite{reseausocial}}. Pour cela, j'ai choisi de développer avec les technologies \aphp\ et \asf. Comme l'utilisation de l'outil issu de \asl\ m'avait beaucoup plu, j'ai tout de suite pensé à postuler dans cette entreprise pour mon stage d'assistant ingénieur. Il s'est avéré que ma connaissance préalable de \asf\ m'a été extrêmement utile pour m'immerger rapidement dans ma mission de stage.

Au delà de l'enjeu technologique, j'ai été vivement attiré par le fait que \asl, en distribuant son outil sous licence libre\footnote{Une licence libre est une licence s'appliquant à une œuvre d'esprit par laquelle l'auteur concède tout ou une partie des droits que lui confère le droit d'auteur.~\cite{licencelibre}}, est acteur majeur de l'\aos\footnote{La désignation \aos\ s'applique aux logiciels donc la licence respecte des critères précisément établis par l'\aosinitiative, c'est-à-dire la possibilité de libre redistribution, d'accès au code source et de travaux dérivés.~\cite{os}}. En effet, l'écosystème du logiciel libre\footnote{Un logiciel libre est un logiciel dont l'utilisation, l'étude, la modification, la duplication et la diffusion sont universellement autorisées sans contrepartie.~\cite{logiciellibre}} est un domaine qui me passionne, dans lequel j'aimerais éventuellement exercer mon futur métier d'ingénieur. J'ai alors pensé qu'ajouter un tel stage à mon parcours professionnel serait une excellente piste pour m'engager dans cette voie.

Ainsi, travailler chez \asl\ a bien été pour moi le fruit d'un réel désir, et non pas d'un choix par défaut.
