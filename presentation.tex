\chapter{Présentation de l'entreprise}

\section{L'entreprise \asensio}

La société \asensio\ a été créée en 1988 par les deux co-fondateurs \apotencier\ et \apascal. Dès ses premiers mois, elle s'est très vite orientée vers les technologies de l'\ainternet. Elle s'inscrit alors comme une véritable agence web interactive, proposant à ses clients un savoir-faire dans tous les métiers du web : développement, expertise technique, \awm, communication et \awd. Aujourd'hui, elle est basée dans la ville de Clichy, située près de Paris.

En 2007, \asensio\ s'est rapprochée d'\aextreme\footnote{À l'occasion du 29ème Grand Prix des \aagencesannee en 2009, \aextreme s'est vu décerné le prix du groupe de communication indépendant de l'année}, un grand groupe indépendant de communication globale\footnote{publicité, marketing, services, web, \textit{packaging}, design, \textit{corporate}}. Cette démarche commerciale, et non capitalistique, distingue bien les deux entités en deux sociétés à part entière. Elle a pour origine un désir des deux parties : d'un côté \aextreme\ souhaitait se rapprocher d'une agence web, et de l'autre, \asensio\ ressentait le besoin d'acquérir de meilleures compétences dans le domaine de la communication.

L'association de \asensio\ et d'\aextreme\ a donné naissance à trois entités commerciales, ou \abusfull :

\begin{description}
	\item[\asl] s'occupe de la partie développement web ;
	\item[\aes] gère tout l'aspect \awm\ et communication ;
	\item[\aesm] propose à ses clients des plans média destinés à doper leur trafic et leurs ventes.
\end{description}

La force de \asensio\ réside donc dans le fait de pouvoir faire travailler ensemble des personnes aux profils de natures très différentes, et cela afin de répondre au mieux aux attentes de ses clients.

Rentable dès ses débuts, \asensio\ dégage un chiffre d'affaires de 6~millions d'euros pour l'année~2009, d'après les estimations actuelles.


\section{La \abufull\ \asl}

Chez \asensio, j'ai travaillé dans la \abufull\ \asl. Dirigée par \apotencier, son cœur de métier est de développer des applications web pour les grands comptes. En effet, ses principaux clients sont Peugeot, EDF, l'Office de Tourisme et des Congrès de Paris, Infogreffe, Evian\dots

Par ailleurs, \asl\ est le créateur d'un outil de travail connaissant aujourd'hui un succès international formidable : le \afm\ \aphp\ \asf\ décrit en section~\ref{section:outils_sf}. Celui-ci apporte à l'entreprise une visibilité grandissante associée à une image de marque. 

\asl\ est constitué de trois pôles aux ambitions différentes :

\begin{description}
	\item[le pôle projet] est composé de chefs de projets, dont le but est s'occuper de toute la partie communication avec le client, de faire l'intermédiaire entre les différents acteurs techniques d'un projet, de rédiger son cahier des charges, de planifier le temps et les ressources qui lui sont affectés ;
	\item[le pôle production] est composé d'intégrateurs, qui ont pour objectif de transformer les maquettes de sites, réalisées par les créateurs d'\aes, en pages web statiques compatibles avec un certain nombre de navigateurs du marché ;
	\item[le pôle développement] se compose d'une équipe de développeurs, dont le but est de rendre dynamiques les pages web produites par le pôle production, en se chargeant de l'implémentation de leur logique.
\end{description}

Dans le cadre de mon stage, j'ai travaillé au pôle développement en tant que développeur.

Il faut noter que certains projets, concernant des sites web évènementiels ou promotionnels par exemple, sont orchestrés par les chefs de projets d'\aes\ et non pas par ceux du pôle projet de \asl. Toutefois, les réalisations techniques sont bien issues des pôles production et développement de ce dernier.


\section{Mon choix de \asl}

TODO
